\hypertarget{ux7f16ux8bd1ux5b9eux4e60ux5c0fux4f5cux4e1a-ux57faux4e8eux8bedux6cd5ux5236ux5bfcux7ffbux8bd1ux7684ux7b80ux6613ux8ba1ux7b97ux5668}{%
\section{编译实习小作业
基于语法制导翻译的简易计算器}\label{ux7f16ux8bd1ux5b9eux4e60ux5c0fux4f5cux4e1a-ux57faux4e8eux8bedux6cd5ux5236ux5bfcux7ffbux8bd1ux7684ux7b80ux6613ux8ba1ux7b97ux5668}}

\hypertarget{ux7f57ux660a-1700010686}{%
\subsection{罗昊 1700010686}\label{ux7f57ux660a-1700010686}}

\hypertarget{ux7b80ux4ecb}{%
\subsection{简介}\label{ux7b80ux4ecb}}

该计算器基于基于语法制导的翻译,由 flex 和 Gnu
Bison创建,支持基本四则运算和赋值运算.

四则运算支持括号和优先级,支持整数运算和浮点数运算,除法运算规则与C语言相同.

支持通过赋初值的方式创建变量,变量名与c语言标识符风格相同,不支持中文变量名.

支持等号串联赋值,类似c语言,例如a=b=3.

支持多语句串联,可用分号或回车作为分隔符.

遇到非法字符,使用未定义的变量,语法错误时会给出相应的错误提示,可继续输入新语句.

退出命令为quit

\hypertarget{ux7f16ux8bd1}{%
\subsection{编译}\label{ux7f16ux8bd1}}

依赖 flex bison g++

测试环境为 flex-2.6.4 , bison-3.0.4 , g++-7.3.0

已写makefile, 直接make即可编译, 输出可执行文件为calc.

\begin{verbatim}
make
./calc
\end{verbatim}

\hypertarget{ux793aux4f8b}{%
\subsection{示例}\label{ux793aux4f8b}}

\begin{verbatim}
1+2* (1+3)
9

2.0 * 3
6.000000

3/  2
1

3./2
1.500000

a=3*-1
-3

a+5
2

b+3
Error encountered: identifier b not defined!

b=a=13.1
13.100000

c=2; c+1
2
3

编译实习=100
Error encountered: Illegal Character!

2+
Error encountered: syntax error

quit
\end{verbatim}

\hypertarget{ux4ee3ux7801ux7ed3ux6784}{%
\subsection{代码结构}\label{ux4ee3ux7801ux7ed3ux6784}}

代码由两个文件组成:calc.l和calc.y

\hypertarget{calc.l}{%
\subsubsection{calc.l}\label{calc.l}}

由三部分组成:\#include 部分,词法规则,附加代码.

第一部分包含了bison生成的y.tab.h,其中有一些类型的定义和token宏的定义.

第二部分为词法规则,主要的token有运算符、整数、浮点数、标识符、分隔符、quit以及含有非法字符的语句.

第三部分为附加代码,只有yywrap()函数. \#\#\# calc.y
由三部分组成:类型定义、token声明和附加声明和定义,语法规则,附加代码

第一部分有

\%code
requires部分会被附加到y.tab.h中,包含需要include的库,自定义的结构体,yy系列函数的声明.
(\%code
requires的作用查阅自\href{http://www.gnu.org/software/bison/manual/bison.html}{Gnu
Bison文档}

\%union部分,定义YYSTYPE为union类型.
(\%union的作用查阅自\href{http://www.gnu.org/software/bison/manual/bison.html}{Gnu
Bison文档}

\%code部分,包含一个全局变量和所需函数的定义.

全局变量为一个作为符号表的unordered\_map,键类型为string,值类型为一个包含type和value的结构体,其中type代表变量是实数还是整数,value是一个union,储存变量的值.

所需函数有用于进行单步运算的calc函数和negative函数,以及用于报错的yyerror函数.

然后是token的声明和类型定义,以及部分非终结符号的类型定义.

接下来\%right 和 \%left 声明运算符的结合性和优先级顺序 (\%right \%left
和 \%prec
的作用和优先顺序查阅自\href{http://www.gnu.org/software/bison/manual/bison.html}{Gnu
Bison文档}

第二部分是语法定义,设置了三个非终结符,stmts expr 和 assignexpr.

stmts是多条语句 expr是单条非赋值语句 assignexpr是单条赋值语句.

第三部分为附加代码,包含main函数.
